\section{The Sorting Problem}

%%%%%%%%%%%%%%%%%%%%
\begin{frame}{Algorithms}
  % \begin{center}
  %   An algorithm is a sequence of operations \\
  %   that transform the input into the output.
  % \end{center}
  \begin{center}
	What is an algorithm? \qquad \pause What is computation?
  \end{center}

  \pause

  \fignocaption{width = 0.50\textwidth}{figs/algorithm-def.pdf}

  \pause
  \vspace{-0.60cm}
  \begin{center}
	\textcolor{red}{Correctness!}
  \end{center}

  \pause

  \begin{description}[Effectiveness:]
	\item[Definiteness:] precisely defined operations
	  \pause
	\item[Finiteness:] termination
	  \pause
	\item[Effectiveness:] a reasonable model; basic operations % RAM {\scriptsize (Random-Access Machine)} model
	  \pause
	  \begin{itemize}
		% \item unrealistic: \texttt{sort} operation
		% \item realistic: arithmetic, data movement, and control
		%   \pause
		\item for sorting: compare, swap
	  \end{itemize}
  \end{description}
\end{frame}
%%%%%%%%%%%%%%%%%%%%
\begin{frame}{Sorting}
  The sorting problem:
  \begin{description}
	\item[Input:] A sequence of $n$ integers $A$:\seq{$a_1, a_2, \cdots, a_n$}.
	\item[Output:] A permutation \seq{$a'_1, a'_2, \ldots, a'_n$} of $A$ \emph{s.t.} $a'_1 \le a'_2 \le \cdots \le a'_n$ {\small (non-decreasing order)}.
  \end{description}

  \[
	3\quad 1\quad 4\quad 2\quad \Longrightarrow 1\quad 2\quad 3\quad 4
  \]

  % sortable
  % A little more formalism: ordering relation ``$<$'' on $A$.

  % \vspace{0.20cm}
  % $\forall a, b, c \in A$,
  % \begin{description}[Transitivity:]
  %   \item[Trichotomy:] $a < b, a = b, a > b$
  %   \item[Transitivity:] $a < b \land b < c \implies a < c$
  % \end{description}
\end{frame}
%%%%%%%%%%%%%%%%%%%%
\begin{frame}{Inversions}
  \[
	A = a_1\quad a_2\quad \ldots\quad a_n.
  \]

  \begin{center}
	If $i < j$ and $a_{i} > a_{j}$, then $(a_i, a_j)$ is an \textcolor{red}{\bf inversion}.\\[8pt] \pause
	\textcolor{blue}{Adjacent} inversion: $(a_i, a_{i+1})$
  \end{center}

  \pause
  \vspace{-0.50cm}

  \begin{columns}
	\column{0.50\textwidth}
	  \fignocaption{width = 0.50\textwidth}{figs/inversions-example.pdf}
	\column{0.50\textwidth}
	{\small
	  \begin{center}
		\#inversions = 3\\
		\#adjacent inversions = 2
	  \end{center}
	}
  \end{columns}

  \pause
  \begin{columns}
	\column{0.50\textwidth}
	  \fignocaption{width = 0.50\textwidth}{figs/inversions-example-nonincreasing.pdf}
	\column{0.50\textwidth}
	{\small
	  \begin{center}
		\#inversions = 3 + 2 + 1 = 6\\
		\#adjacent inversions = 3
	  \end{center}
	}
  \end{columns}

  \pause
  \begin{columns}
	\column{0.50\textwidth}
	  \fignocaption{width = 0.50\textwidth}{figs/inversions-example-nondecreasing.pdf}
	\column{0.50\textwidth}
	{\small
	  \begin{center}
		\#inversions = 0\\
		\#adjacent inversions = 0
	  \end{center}
	}
  \end{columns}
\end{frame}
%%%%%%%%%%%%%%%%%%%%
\begin{frame}{Inversions}
  \begin{center}
	\fbox{\textcolor{blue}{Theorem:} $A$ is sorted $\iff$ $A$ has no adjacent inversions.}
  \end{center}

  \begin{align*}
	\onslide<2->{A \text{ is sorted } &\Longrightarrow A \text{ has no inversions} \\}
	  \onslide<3->{&\Longrightarrow A \text{ has no adjacent inversions}.}
  \end{align*}

  \vspace{-0.50cm}

  \begin{align*}
	\onslide<4->{A \text{ has no adjacent inversions } &\Longrightarrow \forall i \in [1,n-1]: a_{i} \le a_{i+1} \\}
	  \onslide<5->{&\Longrightarrow A \text{ is sorted}.}
  \end{align*}
\end{frame}
%%%%%%%%%%%%%%%%%%%%
