\documentclass[shownotes, xcolor = table]{beamer}

\usepackage{xifthen}
\usepackage{ifthen}
\usepackage{xstring}

\usepackage{verbatim}
\usepackage{listings}

\usepackage{algorithm}
% \usepackage{algorithmic}
\usepackage[noend]{algpseudocode}
\algrenewcommand{\algorithmiccomment}[1]{\hfill$\triangleright$ \textcolor{teal}{#1}}

\usetheme{CambridgeUS} % try Madrid, CambridgeUS
\usecolortheme{beaver} % try beaver, dolphin, seahorse
\usefonttheme{serif}
% \usefonttheme[onlymath]{serif} % try professionalfonts, serif

\usepackage{amsmath, amsfonts, amssymb, mathtools, pifont}
\newcommand{\cmark}{\ding{51}}%
\newcommand{\xmark}{\ding{55}}%
\def\checkmark{\tikz\fill[scale=0.5](0,.35) -- (.25,0) -- (1,.7) -- (.25,.15) -- cycle;} 

\usepackage{graphicx, subcaption}
\usepackage[export]{adjustbox}

% input: sequence of numbers, separated by commas
% output: sequence of numbers, separated by \quad
% See http://tex.stackexchange.com/a/60026/23098
% \usepackage{xstring}
\newcommand{\seq}[1]{%
    \noexpandarg    
    \StrSubstitute{#1}{,}{\;}[\ele]
    \ele}

% needed: \usepackage[export]{adjustbox}
\newcommand{\importikznocaption}[3]{% #1: width, #2: height, #3: tex
\begin{figure}[h!]
    \centering
    \begin{adjustbox}{max totalsize = {#1}{#2}, center}
	  \input{#3}
    \end{adjustbox}
  \end{figure}
}

\usepackage{wasysym}

\usepackage[normalem]{ulem} % strike through text
\newcommand{\soutthick}[1]{%
    \renewcommand{\ULthickness}{2.0pt}%
       \sout{#1}%
    \renewcommand{\ULthickness}{.4pt}% Resetting to ulem default
}

\setbeamersize{text margin left = 2em, text margin right = 1em}
\setbeamercolor{footnote mark}{fg = teal}
\setbeamertemplate{itemize items}[square]
% \setbeamertemplate{itemize items}[default]
\setbeamertemplate{enumerate items}[default]
\newcommand\coloreditem[1]{\item[\textcolor{#1}{\usebeamertemplate{itemize \beameritemnestingprefix item}}]}

\usepackage{tikz}
\usetikzlibrary{mindmap, arrows.meta, shapes, positioning, calc, backgrounds, fit}

\theoremstyle{plain}

\lstdefinestyle{code}{
  language = Java,
  basicstyle = \rmfamily\footnotesize,
  captionpos = t,
  mathescape = true,
  showstringspaces = false,
  keywordstyle = \color{blue}\bfseries,
  commentstyle = \color{teal},
  stringstyle = \color{green}\bfseries,
  numbers = left,
  numberstyle = \tiny,
  % upquote = true, 
  frame = lines,
  breaklines = true
}

% for tables
\usepackage{multirow}
\newcommand{\innercell}[2]{\begin{tabular}{@{}#1@{}}#2\end{tabular}}
\usepackage{hhline}
%%%%%%%%%%%%%% for appendix %%%%%%%%%%%%%%%%
% http://www-ljk.imag.fr/membres/Jerome.Lelong/latex/appendixnumberbeamer.sty
% Reference: http://tex.stackexchange.com/questions/2541/beamer-frame-numbering-in-appendix
\usepackage{appendixnumberbeamer}
% Add total frame count to slides, optional. From Stefan,
% http://www.latex-community.org/forum/viewtopic.php?f=4&t=2173
\expandafter\def\expandafter\insertshorttitle\expandafter{%
  \insertshorttitle\hfill\insertframenumber\,/\,\inserttotalframenumber}
%%%%%%%%%%%%%% for appendix %%%%%%%%%%%%%%%%

% for fig without caption: #1: width/size; #2: fig file
\newcommand{\fignocaption}[2]{
  \begin{figure}[htp]
    \centering
    \includegraphics[#1]{#2}
  \end{figure}
}

% for fig without caption: #1: width/size; #2: fig file; #3: fig caption
\newcommand{\fig}[3]{
  \begin{figure}[htp]
    \centering
      \includegraphics[#1]{#2}
      \caption{#3}
  \end{figure}
}

% for cite: #1: author; #2: conference #3: year
\newcommand{\citeinbeamer}[3]{{\tiny{\textcolor{blue}{[#1@#2'#3]}}}}

\usepackage[backend=bibtex]{biblatex}
\addbibresource{cs-application-report.bib}

\newcommand{\term}[1]{\textcolor{blue}{\scriptsize (#1)}}
\newcommand{\set}[1]{\{#1\}}
\newcommand{\question}[1]{\textcolor{red}{\centerline{#1}}}
\newcommand{\answer}[1]{\textcolor{blue}{\centerline {#1}}}
\newcommand{\alertred}[1]{\textcolor{red}{#1}}
\newcommand{\alertblue}[1]{\textcolor{blue}{#1}}
\newcommand{\todo}[1]{\textcolor{red}{\textbf{TODO:} #1}}
\newcommand{\mathbfblue}[1]{\textcolor{blue}{$\mathbf{#1}$}}

\newcommand{\reporttitle}{Bubble Sort}
%%%%%%%%%%%%%%%%%%%%%%%%%%%%%%%%%%%%%%%%%%%%%%%%%%%%%%%%%%%%%%%%%%%%%%%%%%%%%%%%	
\title[\reporttitle]{\reporttitle}
\subtitle{(What are Algorithms and How to Analyze Algorithms)}

\author[Hengfeng Wei]{Hengfeng Wei}
\titlegraphic{\includegraphics[height = 1.2cm]{figs/nju-logo-purple.png}~\includegraphics[height = 1.2cm]{figs/cs-logo.jpg}}
\institute[ICS@NJU]{Institute of Computer Software\\Nanjing University}
\date{\today}

\AtBeginSection[]{
  \begin{frame}[noframenumbering, plain]
    \frametitle{\reporttitle}
    \tableofcontents[currentsection, sectionstyle=show/shaded, subsectionstyle=hide/hide/hide]
  \end{frame}
}
%%%%%%%%%%%%%%%%%%%%
\begin{document}

\maketitle

\begin{frame}[noframenumbering, plain]
  \frametitle{\reporttitle}
  \tableofcontents[currentsection, sectionstyle=show, subsectionstyle=show/show/hide]
\end{frame}

\section{Sorting}

%%%%%%%%%%%%%%%%%%%%
\begin{frame}{Sorting}
  The sorting problem:
  \begin{center}
	Given a sequence $A$ of \only<3->{\textcolor{blue}{sortable elements}}\only<1-2>{integers},\\
	arrange them in ascending/descending \textcolor<3->{blue}{order}.
  \end{center}

  \pause

  \[
	3\ 1\ 4\ 2 \Longrightarrow 1\ 2\ 3\ 4
  \]

  \pause
  \vspace{0.20cm}

  A little more formalism: ordering relation ``$<$'' on $A$.

  \vspace{0.20cm}
  $\forall a, b, c \in A$,
  \begin{description}[Transitivity:]
	\item[Trichotomy:] $a < b, a = b, a > b$
	\item[Transitivity:] $a < b \land b < c \implies a < c$
  \end{description}
\end{frame}
%%%%%%%%%%%%%%%%%%%%
\begin{frame}{Sorting Algorithms}
  Five important features of (sorting) algorithms:
  \begin{description}[Effectiveness:]
	\item[Input:] an integer array $A$
	\item[Ouput:] $A'$ sorted
	\item[Definiteness:] precisely defined
	  \pause
	\item[\textcolor{red}{Finiteness:}] termination
	  \pause
	\item[\textcolor{red}{Effectiveness:}] operations are sufficiently basic

	  \textcolor{red}{CAS}: compare and swap if out-of-order
  \end{description}

  \pause
  \vspace{0.30cm}

  \begin{center}
	\textcolor{brown}{Roughly, what is computation?}
  \end{center}
\end{frame}
%%%%%%%%%%%%%%%%%%%%
%%%%%%%%%%%%%%%%%%%%
%%%%%%%%%%%%%%%%%%%%
%%%%%%%%%%%%%%%%%%%%
\begin{frame}{Inversions}
  \[
	A = a_1, a_2, \ldots, a_n.
  \]
  

  \begin{center}
	If $i < j$ and $a_{i} > a_{j}$, then $(a_i, a_j)$ is an \textcolor{red}{\bf inversion}.
  \end{center}

  \pause

  \[
	A = 3, 1, 4, 2.
  \]

  \begin{center}
	Inversions: $\textcolor<4->{brown}{(3,1)}, (3,2), \textcolor<4->{brown}{(4,2)}$
  \end{center}

  \pause
  \vspace{0.20cm}

  \begin{center}
	\textcolor{brown}{\bf Adjacent} inversion: $j = i + 1$
  \end{center}
\end{frame}
%%%%%%%%%%%%%%%%%%%%
\begin{frame}{Inversions}
  \begin{center}
	$A$ is sorted $\iff$ $A$ has \emph{no} adjacent inversions.
  \end{center}

  \pause

  \begin{proof}
	\begin{description}
	  \item[$\Longrightarrow:$] No inversions at all.
		\pause
	  \item[$\Longleftarrow:$] $\forall i \in [1,n-1]: a_{i} \le a_{i+1}$.
	\end{description}
  \end{proof}
\end{frame}
%%%%%%%%%%%%%%%%%%%%

\section{Bubble Sort}

%%%%%%%%%%%%%%%%%%%%
\begin{frame}{Bubble Sort: Basic Idea}
  Basic idea: to eliminate all adjacent inversions
  \pause

  % basic idea of bubble sort with overlay

\begin{algorithm}[H]
  \begin{algorithmic}[1]
	\Procedure{BubbleSortOverview}{$A: a_1\; a_2\; \cdots\; a_n$}
	  \Repeat
		\State Pick any $i$		
		\uncover<2->{\Comment{\textcolor{red}{\bf Definiteness!}}}
		  \If{$a_{i} > a_{i+1}$}	\Comment{CAS}
			\State \Call{Swap}{$a_{i}, a_{i+1}$}
		  \EndIf
		  \Until{no adjacent inversions}	
		  \uncover<2->{\Comment{\uncover<3->{\textcolor{red}{\bf Finiteness!}}
		  	\textcolor{red}{\bf Definiteness!}}}
    \EndProcedure
  \end{algorithmic}
\end{algorithm}

\end{frame}
%%%%%%%%%%%%%%%%%%%%
\begin{frame}{Bubble Sort: Definiteness}
  % bubble sort alg

\begin{algorithm}[H]
  \begin{algorithmic}[1]
	\Procedure{BubbleSort}{$A: a_1\; a_2\; \cdots\; a_n$}
	  \Repeat
	  \State \fbox{\uncover<3->{\textcolor{red}{swapped} }\uncover<4->{$\gets$ false}}
		\For{\fbox{\uncover<2->{$i \gets 1 : n - 1$}}}	\Comment{Pick $i$}
		  \If{$a_{i} > a_{i+1}$}	% \Comment{CAS}
			\State \Call{Swap}{$a_{i}, a_{i+1}$}
			\State \fbox{\uncover<5->{swapped $\gets$ true}}
		  \EndIf
		\EndFor
		\Until{\fbox{\only<1-5>{no adjacent inversions}\only<6->{swapped = false}}}
    \EndProcedure
  \end{algorithmic}
\end{algorithm}

\end{frame}
%%%%%%%%%%%%%%%%%%%%
\begin{frame}{Bubble Sort: Example}
  % \begin{center}
  %   \scalebox{0.50}{
  %     \begin{minipage}{0.70\linewidth}
  %   	\input{algs/bubble-sort-alg}
  %     \end{minipage}
  %   }
  % \end{center}

  \begin{center}
	\input{tikz/bubble-sort-example-overlay}
  \end{center}

  \uncover<13->{
	After each ``{\bf for}'' loop, one more element is bubbled up to its final position.
  }
  % \[
  %   5\quad 3\quad 1\quad 6\quad 7\quad 2\quad 4\quad 8
  % \]
\end{frame}
%%%%%%%%%%%%%%%%%%%%
\begin{frame}{Bubble Sort: Finiteness}
  \begin{center}
	\scalebox{0.75}{
	  \begin{minipage}{0.70\linewidth}
		\input{algs/bubble-sort-alg}
	  \end{minipage}
	}
  \end{center}

  The inner ``{\bf for}'' loops:
  \begin{enumerate}[1)]
	\item \textcolor{blue}{$\exists$ loop : no swaps} $\implies$ swapped = false $\implies$ terminates
	  \pause
	\item \textcolor{red}{$\forall$ loop : has swaps} \pause \qquad \textcolor{red}{\bf \textsc{Impossible}!}
  \end{enumerate}
\end{frame}
%%%%%%%%%%%%%%%%%%%%
\begin{frame}{Bubble Sort: Finiteness}
  \begin{center}
	\fbox{Fact: total \#inversions is finite.}
  \end{center}

  \pause
  Effects of \textsc{Swap}{($a_{i}, a_{i+1}$)} on \textcolor{red}{\#inversions}\footnote{Not on \textcolor{blue}{\#adjacent inversions}! Think about it.}:
  \fignocaption{width = 0.50\textwidth}{figs/swap-inversions.pdf}

  \pause
  \begin{center}
	\textcolor{blue}{$-1: (a_{i},a_{i+1})$}\\[3pt] \pause 
	\textcolor{red}{${}+{}0:$} relative order between any other two elements does not change!\\[3pt] \pause
	$(a_k, a_l), (a_m, a_n), \textcolor{red}{(a_k, a_i), (a_i, a_m)}$
  \end{center}

  \pause
  \begin{center}
	\fbox{$\textsc{Swap}(a_{i}, a_{i+1}) \implies -1 \text{ inversion}$}
  \end{center}
\end{frame}
%%%%%%%%%%%%%%%%%%%%
\begin{frame}{Bubble Sort: Correctness}
  \begin{align*}
	\onslide<1->{\text{\textcolor{blue}{\bf Finiteness }}}
	  \onslide<2->{&\implies \exists \text{ loop : no swaps} \\[3pt]}
	  \onslide<3->{&\implies A \text{ has no adjacent inversions any more} \\[3pt]}
	  \onslide<4->{&\implies A \text{ is already sorted.}}
  \end{align*}
\end{frame}
%%%%%%%%%%%%%%%%%%%%
\begin{frame}{Optimizing Bubble Sort (I)\footnote{See Appendix for ``Optimizing Bubble Sort (II)''.}}
  After each ``{\bf for}'' loop, one more element is bubbled up to its final position.

  % mfp: maximal in its final position

\begin{algorithm}[H]
  \begin{algorithmic}[1]
	\Procedure{BubbleSort}{$A: a_1\; a_2\; \cdots\; a_n$}
	  \State $n \gets \text{len}(A)$
	  \Repeat
		\State swapped $\gets$ false
		\For{$i \gets 1 : n - 1$}	% \Comment{Pick $i$}
		  \If{$a_{i} > a_{i+1}$}	% \Comment{CAS}
			\State \Call{Swap}{$a_{i}, a_{i+1}$}
			\State swapped $\gets$ true
		  \EndIf
		\EndFor
		\State $n \gets n - 1$	\Comment{One maximal bubbles up}
		\Until{swapped = false}
    \EndProcedure
  \end{algorithmic}
\end{algorithm}

\end{frame}
%%%%%%%%%%%%%%%%%%%%
\section{Analysis}

%%%%%%%%%%%%%%%%%%%%
\begin{frame}{Time Complexity of Bubble Sort}
  \begin{itemize}
	\setlength{\itemsep}{10pt}
	\item Finiteness is NOT enough $\implies$ Quantitative finiteness
	  \pause
	\item Time on real computers varies $\implies$ \#Ops on RAM model:
	  \vspace{5pt}
	  \pause
	  \begin{description}
		\setlength{\itemsep}{5pt}
		\item[$|P|:$] \#Passes \hfill (the ``{\bf for}'' loops)
		\item[$|C|:$] \#Comparisons \hfill ({\bf if} $a_{i-1} > a_{i}$)
		\item[$|S|:$] \#Swaps  \hfill ($\textsc{Swap}(a_{i-1}, a_{i})$)
	  \end{description}
	  \pause
	\item Different inputs $\implies$ different execution time:
	  \vspace{5pt}
	  \pause
	  \begin{itemize}
		\item Best-case, worst-case, and average-case analysis
	  \end{itemize}
\end{itemize}
\end{frame}
%%%%%%%%%%%%%%%%%%%%
\begin{frame}{Best-case and Worst-case Analysis}
  \uncover<2->{
  \[
	\text{\textcolor{blue}{Best-case:} } \seq{1,2,3,4,5,6,7,8}
  \]}
  
  \begin{displaymath}
	\begin{array}{lll}
	  & \hfill \textcolor{blue}{\frac{\text{Best-case:}}{\uncover<2->{\text{ascendingly sorted}}}} \hfill 
	  & \hfill \textcolor{red}{\frac{\text{Worst-case:}}{\uncover<4->{\text{descendingly sorted}}}} \\[10pt]
	  |P| & = (\uncover<3->{\min: 1,} & \uncover<5->{\max: n}); \\[6pt]
	  |C| & = (\uncover<3->{\min: n-1,} & \uncover<5->{\max: \frac{n^2 - n}{2}}); \\[6pt]
	  |S| & = (\uncover<3->{\min: 0,} & \uncover<5->{\max: \frac{n^2 - n}{2}}).
	\end{array}
  \end{displaymath}

  \uncover<4->{
  \[
	\text{\textcolor{red}{Worst-case: }} \seq{8,7,6,5,4,3,2,1}
  \]}
\end{frame}
%%%%%%%%%%%%%%%%%%%%
\begin{frame}{$|S|:$ \#Swaps (Average Analysis)}
  Assumptions on inputs:
  \begin{enumerate}
	\item The input is a random permutation
	\item All numbers are distinct
  \end{enumerate}

  \pause

  \begin{center}
	\fbox{$\textsc{Swap}(a_{i}, a_{i+1}) \implies -1 \text{ inversion}$}
  \end{center}

  \pause 

  \begin{center}
	\textcolor{red}{\bf Question:} What is the expected \#\text{inversions}?
  \end{center}
\end{frame}
%%%%%%%%%%%%%%%%%%%%
\begin{frame}{$|S|:$ \#Swaps (Average Analysis)}
  \begin{align*}
	\onslide<1->{I_{ij} &: \text{indicator of inversion } (a_i, a_j) \\[3pt]}
	\onslide<2->{X &= \sum_{j} \sum_{i<j} I_{ij} \\[3pt]}
	\onslide<3->{E(X) &= E(\sum_{j} \sum_{i<j} I_{ij}) = \sum_{j} \sum_{i<j} E(I_{ij}) \\[3pt]}
	\onslide<4->{E(I_{ij}) &= P\set{I_{ij}} = \frac{1}{2} \\[3pt]}
	\onslide<5->{E(X) &= \sum_{j} \sum_{i<j} \frac{1}{2} = \binom{n}{2} \cdot \frac{1}{2} = \frac{n(n-1)}{4}}
  \end{align*}
\end{frame}
%%%%%%%%%%%%%%%%%%%%
%%%%%%%%%%%%%%%%%%%%
%%%%%%%%%%%%%%%%%%%%
%%%%%%%%%%%%%%%%%%%%
%%%%%%%%%%%%%%%%%%%%


\end{document}