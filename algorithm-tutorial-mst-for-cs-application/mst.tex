\documentclass[shownotes, xcolor = table]{beamer}

\usepackage{xifthen}
\usepackage{verbatim}

\usetheme{CambridgeUS} % try Madrid
\usecolortheme{beaver} % try beaver, dolphin, seahorse
\usefonttheme[onlymath]{serif} % try "professionalfonts"

\usepackage{amsmath, amsfonts, amssymb, mathtools, pifont}
\newcommand{\cmark}{\ding{51}}%
\newcommand{\xmark}{\ding{55}}%
\def\checkmark{\tikz\fill[scale=0.5](0,.35) -- (.25,0) -- (1,.7) -- (.25,.15) -- cycle;} 

\usepackage{graphicx, subcaption}
\usepackage[export]{adjustbox}

\usepackage[framemethod=TikZ]{mdframed}

\usepackage[normalem]{ulem} % strike through text
\newcommand{\soutthick}[1]{%
    \renewcommand{\ULthickness}{2.0pt}%
       \sout{#1}%
    \renewcommand{\ULthickness}{.4pt}% Resetting to ulem default
}

\setbeamersize{text margin left = 2em, text margin right = 1em}
\setbeamercolor{footnote mark}{fg = teal}
\setbeamertemplate{itemize items}[default]
\setbeamertemplate{enumerate items}[default]

\usepackage{tikz}
\usetikzlibrary{arrows.meta, shapes, positioning, calc, backgrounds, fit}

\theoremstyle{plain}

% for tables
\usepackage{multirow}
\newcommand{\innercell}[2]{\begin{tabular}{@{}#1@{}}#2\end{tabular}}
\usepackage{hhline}
%%%%%%%%%%%%%% for appendix %%%%%%%%%%%%%%%%
% http://www-ljk.imag.fr/membres/Jerome.Lelong/latex/appendixnumberbeamer.sty
% Reference: http://tex.stackexchange.com/questions/2541/beamer-frame-numbering-in-appendix
\usepackage{appendixnumberbeamer}
% Add total frame count to slides, optional. From Stefan,
% http://www.latex-community.org/forum/viewtopic.php?f=4&t=2173
\expandafter\def\expandafter\insertshorttitle\expandafter{%
  \insertshorttitle\hfill\insertframenumber\,/\,\inserttotalframenumber}
%%%%%%%%%%%%%% for appendix %%%%%%%%%%%%%%%%

% for fig without caption: #1: width/size; #2: fig file
\newcommand{\fignocaption}[2]{
  \begin{figure}[htp]
    \centering
    \includegraphics[#1]{#2}
  \end{figure}
}

% for fig without caption: #1: width/size; #2: fig file; #3: fig caption
\newcommand{\fig}[3]{
  \begin{figure}[htp]
    \centering
      \includegraphics[#1]{#2}
      \caption{#3}
  \end{figure}
}

% for cite: #1: author; #2: conference #3: year
\newcommand{\citeinbeamer}[3]{{\tiny{\textcolor{blue}{[#1@#2'#3]}}}}

\usepackage[backend=bibtex]{biblatex}
\addbibresource{cs-application-report.bib}

\newcommand{\term}[1]{\textcolor{blue}{\scriptsize (#1)}}
\newcommand{\set}[1]{\{#1\}}
\newcommand{\question}[1]{\textcolor{red}{\centerline{#1}}}
\newcommand{\answer}[1]{\textcolor{blue}{\centerline {#1}}}
\newcommand{\alertred}[1]{\textcolor{red}{#1}}
\newcommand{\alertblue}[1]{\textcolor{blue}{#1}}
\newcommand{\todo}[1]{\textcolor{red}{\textbf{TODO:} #1}}
\newcommand{\mathbfblue}[1]{\textcolor{blue}{$\mathbf{#1}$}}

\newcommand{\reporttitle}{Minimum Spanning Trees}
%%%%%%%%%%%%%%%%%%%%%%%%%%%%%%%%%%%%%%%%%%%%%%%%%%%%%%%%%%%%%%%%%%%%%%%%%%%%%%%%	
\title[\reporttitle]{\reporttitle}
\subtitle{}

\author[Hengfeng Wei]{Hengfeng Wei}
\titlegraphic{\includegraphics[height = 1.2cm]{figs/nju-logo-purple.png}~\includegraphics[height = 1.2cm]{figs/cs-logo.jpg}}
\institute[ICS@NJU]{Institute of Computer Software\\Nanjing University}
\date{\today}

\AtBeginSection[]{
  \begin{frame}[noframenumbering, plain]
    \frametitle{\reporttitle}
    \tableofcontents[currentsection, sectionstyle=show/shaded, subsectionstyle=hide/hide/hide]
  \end{frame}
}
%%%%%%%%%%%%%%%%%%%%
\begin{document}

% mdf: mdframed; #1: frame color; #2: frame title color; #3: frame title; #4: text
\newcommand{\mdf}[4]{
\begin{mdframed}[frametitle={
  \tikz[baseline = (current bounding box.east), outer sep = 0pt]
  \node[anchor = east, rectangle, fill = #1!20, font = \small]{\strut \textcolor{#2}{#3}};},
  innertopmargin = 2pt, linecolor = #1!20, linewidth = 2pt, topline = true,
  frametitleaboveskip=\dimexpr-\ht\strutbox\relax]
  #4
\end{mdframed}
}

\maketitle

\begin{frame}[noframenumbering, plain]
  \frametitle{\reporttitle}
  \tableofcontents[currentsection, sectionstyle=show, subsectionstyle=show/show/hide]
\end{frame}

\section{The MST Problem}	\label{section:mst-problem}

%%%%%%%%%%%%%%%%%%%%
\begin{frame}{Minimum Spanning Tree}
  $G = (V, E)$: connected, undirected, weighted graph ($w(e)$)

  \importikznocaption{0.40\textwidth}{0.80\textwidth}{tikz-in-beamer/spanning-tree-example-overlay.tex}
  
  \pause
  \vspace{0.30cm}

  Spanning tree $T = (V, E' \subseteq E)$: connected, acyclic \pause ($\Rightarrow n-1$ edges)

  \vspace{0.20cm}
  \uncover<6->{
	\[
	  w(T) = \sum\limits_{e \in E'} w(e)
	\]
  }
\end{frame}
%%%%%%%%%%%%%%%%%%%%
\begin{frame}{Minimum Spanning Tree}
  % \begin{center}
  %   MST: Mimimize $w(T)$ over all possible STs
  % \end{center}

  \[
	\text{MST:} \quad \mathop{\mathrm{arg\,min}}_{T} \; w(T)
  \]

  \importikznocaption{0.60\textwidth}{0.80\textwidth}{tikz-in-beamer/mst-example-overlay.tex}
\end{frame}
%%%%%%%%%%%%%%%%%%%%
\begin{frame}{A Simple Property}
  \fignocaption{width = 0.45\textwidth}{figs/mst-simple-property.pdf}

  \begin{center}
	\textcolor{red}{\bf Cut:} $V = (S, V \setminus S)$
  \end{center}

  \pause

  \begin{enumerate}
	\item MST in each connected component
	\item $ce$: \textcolor{brown}{a} \textcolor{red}{\bf lightest} edge across cut
  \end{enumerate}

  \pause

  \begin{center}
	\textcolor{brown}{\bf Copy\&Paste Argument; Exchange Argument}
  \end{center}
\end{frame}
%%%%%%%%%%%%%%%%%%%%
\begin{frame}{A Wrong Divide \& Conquer Algorithm}
  \begin{description}
    \item[Input:] $G = (V, E, w)$
    \item[Divide:] $V = (S, V \setminus S)$; $\large| |S| - |V \setminus S| \large| \le 1$
    \item<2->[Conquer:] $T_1$: an MST of $S$; $T_2$: an MST of $V \setminus S$
    \item<3->[Combine:] $T_{1} + T_{2} + \set{e}$: $e$ is a lightest edge across $(S, V \setminus S)$
  \end{description}

  \pause
  \vspace{0.50cm}

  \fignocaption{width = 0.30\textwidth}{figs/divide-conquer-mst-counterexample.pdf}
\end{frame}
%%%%%%%%%%%%%%%%%%%%
\begin{frame}{A Wrong Algorithm}
  \begin{alertblock}{What is wrong?}
	\fignocaption{width = 0.30\textwidth}{figs/divide-conquer-mst-counterexample.pdf}

	\centerline{\textcolor{brown}{The edges \textcolor{blue}{$bc$ and $ad$} do \textcolor{red}{\bf not} belong to any MST.}}
  \end{alertblock}

  \pause
  \vspace{0.50cm}

  \begin{alertblock}{What if:}
	\centerline{\textcolor{red}{Invariant:} Manages a set of edges $X$ which is a subset of \textcolor{red}{\bf some} MST.}
  \end{alertblock}
\end{frame}
%%%%%%%%%%%%%%%%%%%%

\section{The Generic MST Algorithm}	\label{section:mst-generic-alg}

%%%%%%%%%%%%%%%%%%%%
\begin{frame}{The Generic MST Algorithm}
  \begin{description}
	\setlength{\itemsep}{8pt}
	\item<1->[Overview:] Grow the MST one edge at a time
	\item<2->[State:] Manage a set of edges $X$
	\item<3->[\textcolor{red}{\bf Invariant:}] Prior to each iteration, $X$ is a subset of some MST
	\item<4->[Iteration:] Pick \only<4>{an edge}\only<5->{a \textcolor{red}{\bf safe} edge} $e$ s.t. \\ $X \cup \set{e}$ is also a subset of some MST
  \end{description}
\end{frame}
%%%%%%%%%%%%%%%%%%%%
\begin{frame}[fragile]{The Generic MST Algorithm}
  

  \begin{proof}
	\begin{description}[Initialization:]
	  \item[Initialization:]
	  \item[Maintenance:] 
	  \item[Termination:] 
	\end{description}
  \end{proof}
\end{frame}
%%%%%%%%%%%%%%%%%%%%
\begin{frame}{The Cut Property}
  \begin{block}{Cut Property}
    \begin{itemize}
	  \item Graph $G = (V, E)$; $X$ is part of an MST.
	  \item A cut $(S, V \setminus S)$ \textcolor{red}{\bf respecting} $X$ ($X$ does not cross
	  $(S, V \setminus S)$)
	  \item Let $e$ be a lightest edge across $(S, V \setminus S)$
	\end{itemize}
	Then, $X + \set{e}$ is part of some MST $T$.
  \end{block}

  \begin{exampleblock}{}
  \end{exampleblock}
\end{frame}
%%%%%%%%%%%%%%%%%%%%
\begin{frame}{The Cut Property}
  \fignocaption{width = 0.40\textwidth}{figs/cut-property-no-name.pdf}

  \begin{proof}
    Basic idea: $e \notin T \Rightarrow e \in T'$.
	\begin{itemize}
	  \item $T + \set{e}$ to construct a cycle $C$
	  \item $\exists e'\in C$ such that $e'$ across the cut; $w(e')
	  \geq w(e)$
	  \item $T' = T + \set{e} - \set{e'}$
	  \item $w(T') \le w(T) \Rightarrow w(T') = w(T) \Rightarrow T' \textrm{ is
	  an MST}$
	\end{itemize}
  \end{proof}
\end{frame}
%%%%%%%%%%%%%%%%%%%%

\section{Kruskal's and Prim's Algorithms}	\label{section:mst-algs}

%%%%%%%%%%%%%%%%%%%%
\begin{frame}[fragile]{Kruskal's Algorithm}
  \begin{lstlisting}[style = code]
  sort (non-descreasingly) the edges $E$

  $X = \emptyset$
  for $e \in E$ in non-descreasing order
    if $X \cup \set{e}$ does not produce cycle
       $X \gets X \cup \set{e}$
  \end{lstlisting}

  \importikznocaption{0.45\textwidth}{0.70\textwidth}{tikz-in-beamer/mst-kruskal-example-overlay.tex}
\end{frame}
%%%%%%%%%%%%%%%%%%%%
\begin{frame}[fragile]{Kruskal's Algorithm}
  \begin{description}
	\item[State:] forest $\triangleq$ a collection of connected components
	\item[Ops:] on connected components
	  \begin{itemize}
		\item cycle detection
		\item union two CCs
	  \end{itemize}
  \end{description}

  \pause

  \begin{center}
	Using the \textcolor{red}{\bf disjoint-set} data structure.
  \end{center}
\end{frame}
%%%%%%%%%%%%%%%%%%%%
\begin{frame}[fragile]{Prim's Algorithm}
  \begin{lstlisting}[style = code]
  $X = \emptyset$
  $S = \set{s}$    // pick any $s \in V$
  $R = V \setminus S$
  while $R \neq \emptyset$
    $e = (u,v) \gets$ a lightest edge across $(S, R)$ 
    $X \gets X \cup \set{e}$
    $S \gets S \cup \set{u} \quad R \gets R \setminus \set{v}$
  \end{lstlisting}

  \importikznocaption{0.45\textwidth}{0.70\textwidth}{tikz-in-beamer/mst-prime-example-overlay.tex}
\end{frame}
%%%%%%%%%%%%%%%%%%%%
\begin{frame}[fragile]{Prim's Algorithm}
  \begin{description}
	\item[State:] a growing tree (CC)
	\item[Op:] identifying a lightest edge
  \end{description}

  \pause

  \begin{center}
	Using the \textcolor{red}{\bf priority-queue (min-heap)} data structure.
  \end{center}
\end{frame}
%%%%%%%%%%%%%%%%%%%%
%%%%%%%%%%%%%%%%%%%%
%%%%%%%%%%%%%%%%%%%%
%%%%%%%%%%%%%%%%%%%%
%%%%%%%%%%%%%%%%%%%%


\end{document}